This page is in progress with more information to be added.



C\+O\+N\+T\+I\+N\+U\+O\+U\+S-\/\+T\+I\+ME A\+PI


\begin{DoxyCode}{0}
\DoxyCodeLine{\textcolor{comment}{\# default}}
\DoxyCodeLine{problem = rom.galerkin.default.Problem<odekeyword>(fomObj,}
\DoxyCodeLine{                                                   decoder,}
\DoxyCodeLine{                                                   romState,}
\DoxyCodeLine{                                                   fomReferenceState)}
\DoxyCodeLine{\textcolor{comment}{\# hyper-\/reduced}}
\DoxyCodeLine{problem = rom.galerkin.hyperreduced.Problem<odekeyword>(fomObj,}
\DoxyCodeLine{                                                        decoder,}
\DoxyCodeLine{                                                        romState,}
\DoxyCodeLine{                                                        fomReferenceState,}
\DoxyCodeLine{                                                        projector)}
\DoxyCodeLine{\textcolor{comment}{\# masked}}
\DoxyCodeLine{problem = rom.galerkin.masked.Problem<odekeyword>(fomObj,}
\DoxyCodeLine{                                                  decoder,}
\DoxyCodeLine{                                                  romState,}
\DoxyCodeLine{                                                  fomReferenceState,}
\DoxyCodeLine{                                                  masker,}
\DoxyCodeLine{                                                  projector)}
\end{DoxyCode}


In the code snippets above, {\ttfamily odekeyword} is needed to identify which time-\/stepping scheme to use, and is one of\+: {\ttfamily Forward\+Euler, R\+K4, A\+B2, Backward\+Euler, B\+D\+F2} (these are thos currently supported, but more schemes will be added overtime). ~\newline
 For example\+:


\begin{DoxyCode}{0}
\DoxyCodeLine{problem = rom.galerkin.default.ProblemRK4(...)}
\end{DoxyCode}


~\newline




D\+I\+S\+C\+R\+E\+T\+E-\/\+T\+I\+ME A\+PI


\begin{DoxyCode}{0}
\DoxyCodeLine{\textcolor{comment}{\# default}}
\DoxyCodeLine{problem = rom.galerkin.default.ProblemDiscreteTime<Two/Three>States(appObj,}
\DoxyCodeLine{                                                                    decoder,}
\DoxyCodeLine{                                                                    romState,}
\DoxyCodeLine{                                                                    fomReferenceState)}
\DoxyCodeLine{\textcolor{comment}{\# hyper-\/reduced}}
\DoxyCodeLine{problem = rom.galerkin.hyperreduced.ProblemDiscreteTime<Two/Three>States(appObj,}
\DoxyCodeLine{                                                                         decoder,}
\DoxyCodeLine{                                                                         romState,}
\DoxyCodeLine{                                                                         fomReferenceState,}
\DoxyCodeLine{                                                                         projector)}
\DoxyCodeLine{\textcolor{comment}{\# masked}}
\DoxyCodeLine{problem = rom.galerkin.masked.ProblemDiscreteTime<Two/Three>States(appObj,}
\DoxyCodeLine{                                                                   decoder,}
\DoxyCodeLine{                                                                   romState,}
\DoxyCodeLine{                                                                   fomReferenceState,}
\DoxyCodeLine{                                                                   masker,}
\DoxyCodeLine{                                                                   projector)}
\end{DoxyCode}


For the discrete-\/time A\+PI, when creating the problem one needs to choose the number of states needed for the stencil. Currently, we support either two or three. So when creating the problem, regardless of the problem type, you need to either use {\ttfamily Two} or {\ttfamily Three}. ~\newline
 For example, to create a default Galerkin for the discrete-\/time A\+PI with two stencil states, one would do\+:


\begin{DoxyCode}{0}
\DoxyCodeLine{problem = rom.galerkin.masked.ProblemDiscreteTimeTwoStates(...)}
\end{DoxyCode}
 