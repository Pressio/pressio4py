This page is in progress with more information to be added.



C\+O\+N\+T\+I\+N\+U\+O\+U\+S-\/\+T\+I\+ME A\+PI


\begin{DoxyCode}{0}
\DoxyCodeLine{\textcolor{comment}{\# default}}
\DoxyCodeLine{problem = rom.lspg.unsteady.default.Problem<odekeyword>(appObj,}
\DoxyCodeLine{                                                        decoder,}
\DoxyCodeLine{                                                        romState,}
\DoxyCodeLine{                                                        fomReferenceState)}
\DoxyCodeLine{}
\DoxyCodeLine{}
\DoxyCodeLine{problem = rom.lspg.unsteady.hyperreduced.Problem<odekeyword>(appObj,}
\DoxyCodeLine{                                                             decoder,}
\DoxyCodeLine{                                                             romState,}
\DoxyCodeLine{                                                             fomReferenceState,}
\DoxyCodeLine{                                                             sampleMeshMappingIndices)}
\DoxyCodeLine{}
\DoxyCodeLine{\textcolor{comment}{\# masked}}
\DoxyCodeLine{problem = rom.lspg.unsteady.masked.Problem<odekeyword>(appObj,}
\DoxyCodeLine{                                                       decoder,}
\DoxyCodeLine{                                                       romState,}
\DoxyCodeLine{                                                       fomReferenceState,}
\DoxyCodeLine{                                                       masker)}
\end{DoxyCode}


In the code snippets above, {\ttfamily odekeyword} is needed to identify which time-\/stepping scheme to use, and is one of\+: {\ttfamily Euler, B\+D\+F2} (these are thos currently supported, but more schemes will be added overtime). Note that L\+S\+PG only makes sense for implicit time stepping. ~\newline
 For example\+:


\begin{DoxyCode}{0}
\DoxyCodeLine{problem = rom.lspg.default.ProblemEuler(...)}
\end{DoxyCode}


~\newline




D\+I\+S\+C\+R\+E\+T\+E-\/\+T\+I\+ME A\+PI


\begin{DoxyCode}{0}
\DoxyCodeLine{\textcolor{comment}{\# default or hyper-\/reduced}}
\DoxyCodeLine{\textcolor{comment}{\# (for LSPG with discrete-\/time API, there is no underlying implementation}}
\DoxyCodeLine{\textcolor{comment}{\# difference between default or hyper-\/reduced since user assembles operators directly)}}
\DoxyCodeLine{problem = rom.lspg.unsteady.ProblemDiscreteTime<Two/Three>States(appObj,}
\DoxyCodeLine{                                                                 decoder,}
\DoxyCodeLine{                                                                 romState,}
\DoxyCodeLine{                                                                 fomReferenceState)}
\DoxyCodeLine{}
\DoxyCodeLine{\textcolor{comment}{\# masked}}
\DoxyCodeLine{problem = rom.lspg.unsteady.masked.ProblemDiscreteTime<Two/Three>States(appObj,}
\DoxyCodeLine{                                                                        decoder,}
\DoxyCodeLine{                                                                        romState,}
\DoxyCodeLine{                                                                        fomReferenceState,}
\DoxyCodeLine{                                                                        masker)}
\end{DoxyCode}


For the discrete-\/time A\+PI, when creating the problem one needs to choose the number of states needed for the stencil. Currently, we support either two or three. So when creating the problem, regardless of the problem type, you need to specify {\ttfamily Two} or {\ttfamily Three}. ~\newline
 For example\+:


\begin{DoxyCode}{0}
\DoxyCodeLine{problem = rom.lspg.unsteady.ProblemDiscreteTimeTwoStates(...)}
\end{DoxyCode}
 